\documentclass{article}

\usepackage{amsmath}
\usepackage{amssymb}

\title{The Girdle Schema}
\author{David Anthony Bau}

\begin{document}

\maketitle

We will attempt to, as much as possible, use the terms used in actual model theory to describe the classes of Girdle.

Every proof the user writes will consist of a sequence of \textbf{steps}.

A \textbf{step} with either be a \textbf{formula}, a \textbf{definition}, or an \textbf{explicit application}.

A \textbf{formula} is a tree of the following operators:

\begin{enumerate}
  \item The logical operators: $\land$, $\lor$, $\implies$, $\iff$, $\lnot$, $\exists$, $\forall$, $=$
  \item Any defined relational operators $R(a, b, c, \dots)$.
  \item Any define functional operators $F(a, b, c, \dots)$.
\end{enumerate}

\end{document}
